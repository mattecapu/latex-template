\usepackage[top=2.5cm, bottom=2.5cm, left=3cm, right=4cm]{geometry}

% encoding of the input files
\usepackage[utf8]{inputenc}
\usepackage[T1]{fontenc}

% typography stuff
\pagestyle{headings}
\linespread{1.3}

% tweaks for typograhpy
\usepackage{microtype}
\usepackage[all]{nowidow}
\usepackage[lastparline]{impnattypo}

% quotes
\usepackage{csquotes}

% tweaks for figure captions
\usepackage[labelfont=bf,textfont=it]{caption}

% tweaks for lists
\usepackage{enumitem}
\setlist{labelindent=\parindent, leftmargin=2\parindent}

% bibs
\usepackage[
  backend=biber,
  citestyle=ieee-alphabetic,
  bibstyle=alphabetic,
  giveninits=true,
  maxnames=99
]{biblatex}

\renewcommand\thesection{\arabic{section}}

% show subsubsections in TOC
\setcounter{tocdepth}{3}
\setcounter{secnumdepth}{3}

% environments for theorems
\usepackage{mathtools, amsthm}
\usepackage{thmtools, thm-restate}

% alignment for display equations
\usepackage{float}

\usepackage{centernot}

\usepackage[thicklines]{cancel}
\renewcommand{\CancelColor}{\color{gray!60}}

% images & plots
\usepackage{tikz}

\usetikzlibrary{cd, arrows, nfold}
\usetikzlibrary{decorations.pathreplacing}

% finer numbering for equations
\numberwithin{equation}{section}

\newenvironment{restheorem}[2][]
	{\restatable[#1]{theorem}{#2}}
	{\endrestatable}
\newenvironment{resproposition}[2][]
	{\restatable[#1]{proposition}{#2}}
	{\endrestatable}
\newenvironment{rescorollary}[2][]
	{\restatable[#1]{corollary}{#2}}
	{\endrestatable}
\newenvironment{reslemma}[2][]
	{\restatable[#1]{lemma}{#2}}
	{\endrestatable}
\newenvironment{resdefinition}[2][]
	{\restatable[#1]{definition}{#2}}
	{\endrestatable}

% % enable Appendices environment
% \usepackage[toc, page]{appendix}

% break urls (in bibliography too)
\usepackage{xurl}
% enable hyperlinks in the doc
\usepackage{hyperref}
\hypersetup{
	colorlinks=true,       % false: boxed links; true: colored links
	linkcolor=blue,        % color of internal links
	filecolor=blue,        % color of file links
	urlcolor=blue,         % color of external links
	citecolor=blue,        % color of citation links
	pdfborder={0 0 0},      % no border around links
	breaklinks=true,
	bookmarks=true,
	bookmarksopen=true,
	bookmarksnumbered=true,
	bookmarksdepth=3
}
\setcounter{tocdepth}{2}

% automatic reference naming
\usepackage[nameinlink,capitalize,noabbrev]{cleveref}

\usepackage[english]{babel}

%  environments defs
\theoremstyle{plain}
\newtheorem{theorem}{Theorem}[section]
\newtheorem{proposition}[theorem]{Proposition}
\newtheorem{corollary}{Corollary}[theorem]
\newtheorem{lemma}[theorem]{Lemma}
\newtheorem{fact}[theorem]{Fact}
\newtheorem{property}[theorem]{Property}
\newtheorem{axiom}{Axiom}
\newtheorem{problem}{Problem}
\newtheorem{question}{Question}
\newtheorem{claim}{Claim}

\theoremstyle{definition}
\newtheorem{remark}[theorem]{Remark}
\newtheorem{definition}[theorem]{Definition}
\newtheorem{notation}[theorem]{Notation}
\newtheorem{construction}[theorem]{Construction}
\newtheorem{justunpacking}[theorem]{Unpacking}
\newtheorem{example}[theorem]{Example}
\newtheorem{counterexample}[theorem]{Counterexample}
\newtheorem{nonexample}[theorem]{Non-Example}
\newtheorem{impexample}[theorem]{Important Example}
\newtheorem{exercise}[theorem]{Exercise}

% Redefine the theorem environment to accept an argument
\makeatletter
\newenvironment{unpacking}[1]
  {\renewcommand\thejustunpacking{\cref{#1}}\justunpacking}
  {\endjustunpacking}
\makeatother

\crefname{unpacking}{Unpacking}{Unpacking}
\crefname{diagram}{Diagram}{Diagrams}
\crefname{equation}{Equation}{Equations}

% starred environments defs
\theoremstyle{plain}
\newtheorem*{theorem*}{Theorem}
\newtheorem*{proposition*}{Proposition}
\newtheorem*{corollary*}{Corollary}
\newtheorem*{lemma*}{Lemma}
\newtheorem*{property*}{Property}
\newtheorem*{axiom*}{Axiom}
\newtheorem*{problem*}{Problem}
\newtheorem*{question*}{Question}
\newtheorem*{claim*}{Claim}

\theoremstyle{definition}
\newtheorem*{definition*}{Definition}
\newtheorem*{unpacking*}{Unpacking}
\newtheorem*{example*}{Example}
\newtheorem*{counterexample*}{Counterexample}
\newtheorem*{nonexample*}{Non-Example}
\newtheorem*{impexample*}{Important Example}
\newtheorem*{exercise*}{Exercise}
\newtheorem*{construction*}{Construction}
\newtheorem*{remark*}{Remark}

% symbols
% \usepackage{mathrsfs}
% \usepackage{amssymb, amsfonts}
% \usepackage{textcomp}
% \usepackage{amsmath}
\usepackage{fonttable}
\usepackage{stmaryrd}
% \usepackage{wasysym}
\usepackage{tabularx}
\usepackage{expl3}
\usepackage{xparse, xpatch}
\usepackage{stackengine, stackrel, old-arrows}
\usepackage{manfnt}
\usepackage[matha]{mathabx}

\usepackage{newpxtext}
\usepackage{newpxmath}
\usepackage[bb=px, oldbold]{mathalpha}

\usepackage{quiver}

% todos
\newcommand{\matteo}[1]{\textcolor{red}{#1}}

% scaled & centered figures
\newcommand{\sctikzfig}[2][.8]{\begin{center}\scalebox{#1}{\tikzfig{#2}}\end{center}}

% wide equations (for eqs in enums)
\newcommand{\makeitwide}{\displayindent0pt\displaywidth\textwidth}

\usepackage{scrextend}

\AddToHook{env/equation/before}{%
	\makeitwide%
	\begin{addmargin*}[-3cm]{-2.25cm}
	\setlength{\belowdisplayshortskip}{10pt}%
}
\AddToHook{env/equation/after}{%
	\end{addmargin*}
}

% aligned equations
\newenvironment{eqalign}{\begin{equation}\begin{aligned}}{\end{aligned}\end{equation}}
\newenvironment{eqalign*}{\begin{equation*}\begin{aligned}}{\end{aligned}\end{equation*}}

\newenvironment{eqcenter}{\begin{equation}\begin{gathered}}{\end{gathered}\end{equation}}
\newenvironment{eqcenter*}{\begin{equation*}\begin{gathered}}{\end{gathered}\end{equation*}}

% disjoint footnotes
\newcommand{\disjointfootnotemark}[1]{\footnotemark[\getrefnumber{#1}]}
\newcommand{\disjointfootnotetext}[1]{%
  \addtocounter{footnote}{1}%
  \addtocounter{Hfootnote}{1}%
  \footnotetext{#1}%
}

% overset without decreasing font size
\newcommand{\Overset}[2]{%
  \mathop{#2}\limits^{\vbox to -.1ex{%
  \kern -1.8ex\hbox{$#1$}\vss}}%
}
% underset without decreasing font size
\newcommand{\Underset}[2]{%
  \mathop{#2}\limits_{\vbox to .1ex{%
  \kern -.6ex\hbox{$#1$}\vss}}%
}

% fat semicolon
\newcommand{\concat}{\fatsemi}

% hyphen for math mode
\mathchardef\dash="2D

% defined term
\newcommand{\defining}[1]{\textbf{#1}}

% subject of a thesis
\renewcommand{\th}[1]{\overset{th}{#1}}

% e costant
\newcommand{\e}{\mathrm{e}}

% exp
\renewcommand{\exp}{\operatorname{exp}}

% cotangent
\newcommand{\cotan}{\operatorname{cotan}}

% argmin
\newcommand{\argmin}{\operatorname{argmin}}

% 'does not imply' symbol
\newcommand{\nimplies}{\centernot\implies}

% implications in the opposite direction
\newcommand{\implied}{\Longleftarrow}
\newcommand{\nimplied}{\centernot\implied}

% logical implication
\newcommand{\limp}{\rightarrow}
\newcommand{\liff}{\leftrightarrow}

% inhabitation for types
\newcommand{\tin}{\!:\!}

% inverses of \to
\newcommand{\ot}{\leftarrow}
\newcommand{\from}{\ot}

% long version of \to
\newcommand{\longto}{\longrightarrow}

% inverse of \mapsto
% \newcommand{\mapsfrom}{\mathrel{\reflectbox{\ensuremath{\mapsto}}}}
% \newcommand{\longmapsfrom}{\mathrel{\reflectbox{\ensuremath{\longmapsto}}}}

% inclusion
\newcommand{\into}{\hookrightarrow}
\newcommand{\inot}{\hookleftarrow}
\newcommand{\monoto}{\rightarrowtail}

% surjection
\newcommand{\onto}{\twoheadrightarrow}
\newcommand{\epito}{\twoheadrightarrow}

% iso arrows
\newcommand{\isoto}{\overset{\sim}\to}
\newcommand{\isolongto}{\overset{\sim}\longto}

% 2-morphisms
\newcommand{\twoto}{\Rightarrow}
\newcommand{\isotwoto}{\overset{\sim}\twoto}
\newcommand{\longtwoto}{\Longrightarrow}
\newcommand{\isolongtwoto}{\overset{\sim}\longtwoto}

% 3-morphisms
\newcommand{\threeto}{\Rrightarrow}

\newcommand{\narrow}[2]{\overset{#1}{#2}}
\newcommand{\nto}[1]{\xrightarrow{#1}}
\newcommand{\nfrom}[1]{\xleftarrow{#1}}
\newcommand{\ninto}[1]{\narrow{#1}{\into}}
\newcommand{\nisoto}[1]{\narrow{#1}{\isolongto}}
\newcommand{\nepito}[1]{\narrow{#1}{\epito}}
\newcommand{\nmonoto}[1]{\narrow{#1}{\monoto}}

% profunctors
\newcommand{\profto}{\stackMath\mathrel{\stackinset{c}{-0.25ex}{c}{0.25ex}{\shortmid}{\to}}}
\newcommand{\longprofto}{\stackMath\mathrel{\stackinset{c}{-0.25ex}{c}{0.25ex}{\shortmid}{\longrightarrow}}}
\newcommand{\nprofto}[1]{\narrow{#1}{\profto}}
\newcommand{\profequalto}{\stackMath\mathrel{\stackinset{c}{-0.15ex}{c}{0.1ex}{\shortmid}{\equalto}}}

% double categories
\newcommand{\tightto}{\longto}
\newcommand{\looseto}{\stackMath\mathrel{\stackinset{c}{-0.15ex}{c}{0.1ex}{\loosemarking}{\longto}}}
\newcommand{\slackto}{\stackMath\mathrel{\stackinset{c}{-0.15ex}{c}{0.1ex}{\slackmarking}{\longto}}}
\newcommand{\loosetwoto}{\stackMath\mathrel{\stackinset{c}{-0.15ex}{c}{0.1ex}{\loosemarking}{\twoto}}}
\newcommand{\ntightto}[1]{\narrow{#1}{\tightto}}
\newcommand{\nlooseto}[1]{\narrow{#1}{\looseto}}
\newcommand{\nslackto}[1]{\narrow{#1}{\slackto}}
\newcommand{\lequalto}{\stackMath\mathrel{\stackinset{c}{-0.15ex}{c}{0.1ex}{\loosemarking}{\equalto}}}
\newcommand{\sequalto}{\stackMath\mathrel{\stackinset{c}{-0.15ex}{c}{0.1ex}{\slackmarking}{\equalto}}}

% optics
\newcommand{\lensto}{\leftrightarrows}
\newcommand{\chartto}{\rightrightarrows}
\newcommand{\chartfrom}{\leftleftarrows}
\newcommand{\equalto}{=\mathrel{\mkern-3mu}=}
\newcommand{\nequalto}[1]{\overset{#1}{\equalto}}
\newcommand{\nlensto}[2]{\overset{#1}{\underset{#2}\lensto}}
\newcommand{\nchartto}[2]{\overset{#1}{\underset{#2}\chartto}}
\newcommand{\nchartfrom}[2]{\overset{#1}{\underset{#2}\chartfrom}}

% constant function
\newcommand{\cost}{\text{cost.}}
\newcommand{\const}{\mathsf{const}}

% locutions
\newcommand{\word}[1]{\quad\text{\underline{#1}}\quad}
\newcommand{\almosteverywhereon}[2][\mu]{{\text{${#1}$-a.e. on ${#2}$}}}
\renewcommand{\ae}{\ \text{a.e.}}
\newcommand{\sse}{\word{iff}}
\newcommand{\means}{\word{means}}
\newcommand{\impl}{\word{implies}}
\newcommand{\fa}{\ \text{f.a.}\;}

% such that
\newcommand{\suchthat}{\,|\,}

% numerical sets
\newcommand{\N}{\mathbb{N}}
\newcommand{\Z}{\mathbb{Z}}
\newcommand{\Q}{\mathbb{Q}}
\newcommand{\R}{\mathbb{R}}
\newcommand{\C}{\mathbb{C}}

% direct sum
\newcommand{\dir}{\oplus}
\newcommand{\bigdir}{\bigoplus}

% operations in an Heyting algebra
\newcommand{\hey}{\Rightarrow}
\newcommand{\bigsup}{\bigvee}
\newcommand{\biginf}{\bigwedge}

% differential
\newcommand{\diff}[1]{\operatorname{d}{#1}}
% jacobian
\newcommand{\jac}{\operatorname{\vv{J}}}

% derivatives
\newcommand{\de}{\mathrm{d}}
\newcommand{\dx}{\de x}
\newcommand{\dt}{\de t}
\newcommand{\ds}{\de s}

\newcommand{\der}[2]{\frac{\de{#1}}{\de{#2}}}
\newcommand{\pder}[2]{\frac{\partial {#1}}{\partial {#2}}}

% second derivatives
\newcommand{\sder}[2]{\frac{\de^2{#1}}{\de{#2}^2}}
\newcommand{\spder}[3]{\frac{\partial^2{#1}}{\partial{#2} \partial{#3}}}
% second derivative on the same coordinate
\newcommand{\sdpder}[2]{\frac{\partial^2{#1}}{\partial{#2}^2}}

% big derivatives
\newcommand{\bigder}[2]{\dfrac{\strut \de{#1}}{\de{#2}}}
\newcommand{\bigpder}[2]{\dfrac{\strut \partial {#1}}{\partial {#2}}}

% big second derivatives
\newcommand{\bigsder}[2]{\dfrac{\strut \de^2 {#1}}{\de{#2}^2}}
\newcommand{\bigspder}[3]{\dfrac{\strut \partial^2 {#1}}{\partial {#2} \partial {#3}}}
% big second derivative on the same coordinate
\newcommand{\bigsdpder}[2]{\dfrac{\strut \partial^2 {#1}}{\partial {#2}^2}}

% left/right applied partial derivatives
\newcommand{\lpartial}{\overset{\leftarrow}\partial}
\newcommand{\rpartial}{\overset{\rightarrow}\partial}

% complex stuff
\newcommand{\conj}[1]{\overline{#1}}
\newcommand{\Arg}{\operatorname{Arg}}
\newcommand{\Res}{\operatorname{Res}}

% sign function
\newcommand{\sign}{\operatorname{}{sgn}}

% convergence
\newcommand{\conv}[1][]{\underset{{#1}}{\longrightarrow}}

% regularity classes
\newcommand{\Cn}{\mathcal{C}}
\newcommand{\Czero}{\Cn^0}
\newcommand{\Cone}{\Cn^1}
\newcommand{\Ctwo}{\Cn^2}
\newcommand{\Cinfty}{\Cn^\infty}

% Lipschitz
\newcommand{\Lip}{\mathrm{Lip}}

% indicator function
\newcommand{\ind}{\vv{1}}

% lenses
\newcommand{\biglens}[2]{
	 \begin{pmatrix}{\vphantom{f_f^f}#1} \\ {\vphantom{f_f^f}#2} \end{pmatrix}
}
\newcommand{\littlelens}[2]{
	 \begin{psmallmatrix}{\vphantom{f}#1} \\ {\vphantom{f}#2} \end{psmallmatrix}
}
\newcommand{\lens}[2]{
  \relax\if@display
	 \biglens{#1}{#2}
  \else
	 \littlelens{#1}{#2}
  \fi
}

% \usepackage{xstring}
% \newcommand{\cat}[1]{
%   \relax
%   \StrLen{#1}[\catarglen]
%   \ifnum\catarglen=1
% 	\mathcal{#1}
%   \else
% 	\mathbf{#1}
%   \fi
% }
\newcommand{\cat}[1]{{\mathbf{#1}}}
\newcommand{\dblcat}[1]{\cat{\mathbfbb #1}}
\newcommand{\trplcat}[1]{\cat{\mathfrak #1}}

\newcommand{\dblSet}{\dblcat{Set}}
\newcommand{\trplSet}{\trplcat{Set}}
\newcommand{\dblCat}{\dblcat{Cat}}

\newcommand{\TwoCat}{{\boldsymbol{2}\dblCat}}
\newcommand{\Bicat}{\trplcat{Bicat}}

\newcommand{\Poly}{\cat{Poly}}

\newcommand{\dist}{\Delta}
\newcommand{\pow}{\mathcal{P}}

\newcommand{\dom}{{\partial_0}}
\newcommand{\cod}{{\partial_1}}
\newcommand{\arr}{{\mathsf{\partial_{01}}}}

\newcommand{\st}{\mathrm{st}}

\newcommand{\Sub}{\mathrm{Sub}}

\newcommand{\eval}{\mathrm{eval}}
\newcommand{\curr}{\mathrm{curr}}

\newcommand{\true}{\mathsf{true}}

\newcommand{\view}{\mathsf{view}}
\newcommand{\play}{\mathsf{play}}
\newcommand{\coplay}{\mathsf{coplay}}

\newcommand{\name}[1]{\lceil #1 \rceil}
\DeclareMathOperator{\argmax}{\mathrm{argmax}}

% finite stuff
\newcommand{\fin}{\mathsf{fin}}

% identity
\newcommand{\identity}{\mathrm{id}}
\newcommand{\id}{\mathrm{id}}

% isomorphism and equivalence symbols
\newcommand{\iso}[1][]{\overset{#1}{\cong}}
\newcommand{\equi}{\simeq}

% F left adjoint to G symbol
\newcommand{\adj}{\dashv}

% categories
\newcommand{\Ob}{\operatorname{Ob}}
\newcommand{\Hom}{\operatorname{Hom}}
\newcommand{\End}{\operatorname{End}}
\newcommand{\Aut}{\operatorname{Aut}}
\newcommand{\Nat}{\operatorname{Nat}}

% Kan extensions
\newcommand{\Lan}{\operatorname{Lan}}
\newcommand{\Ran}{\operatorname{Ran}}
\newcommand{\lan}{\operatorname{lan}}
\newcommand{\ran}{\operatorname{ran}}


% big categories
\newcommand{\Cat}{\cat{Cat}}
\newcommand{\Prof}{\cat{Prof}}

\newcommand{\Set}{\cat{Set}}
\newcommand{\FinSet}{\cat{Set}_\fin}
\newcommand{\Fam}{\cat{Fam}}
\newcommand{\FinFam}{\cat{Fam}_\fin}

\newcommand{\Mon}{\cat{Mon}}
\newcommand{\CMon}{\cat{CMon}}
\newcommand{\Grp}{\cat{Grp}}
\newcommand{\Mod}{\cat{Mod}}
\newcommand{\Ab}{\cat{Ab}}
\newcommand{\Vect}{\cat{Vect}}
\newcommand{\Met}{\cat{Met}}
\newcommand{\Meas}{\cat{Meas}}
\newcommand{\Msbl}{\cat{Msbl}}
\newcommand{\Prob}{\cat{Prob}}
\newcommand{\Euc}{\cat{Euc}}
\newcommand{\Smooth}{\cat{Smooth}}

% dualities
\newcommand{\re}{{\mathsf{re}}}
\newcommand{\co}{{\mathsf{co}}}
\newcommand{\op}{{\mathsf{op}}}
\newcommand{\opre}{{\mathsf{reop}}}
\newcommand{\core}{{\mathsf{reco}}}
\newcommand{\coop}{{\mathsf{coop}}}
\newcommand{\recoop}{{\mathsf{recoop}}}

\newcommand{\rev}{{\mathsf{rev}}}
\newcommand{\lop}{{\mathsf{lop}}}
\newcommand{\iop}{{\mathsf{top}}}
\newcommand{\ltop}{{\mathsf{ltop}}}

\newcommand{\Optic}{\cat{Optic}}
\newcommand{\DLens}{\cat{DLens}}
\newcommand{\DChart}{\cat{DChart}}

\newcommand{\Bun}{\mathrm{Bun}}

\newcommand{\MonCat}{\cat{MonCat}}
\newcommand{\SymMonCat}{\cat{SymMonCat}}
\newcommand{\Cart}{\cat{Cart}}
\newcommand{\Fib}{\cat{Fib}}

\newcommand{\Kl}{\dblcat{Kl}}
\newcommand{\Alg}{\dblcat{Alg}}

\newcommand{\lax}{\mathsf{lx}}
\newcommand{\colax}{\mathsf{cx}}
\newcommand{\pseudo}{\mathsf{ps}}
\newcommand{\strict}{\mathsf{s}}
\newcommand{\cart}{\mathsf{cart}}
\newcommand{\ver}{\mathsf{vert}}
\newcommand{\disp}{\mathsf{disp}}

\newcommand{\VCat}[1]{{#1}\dash\Cat}

\newcommand{\swap}{\mathsf{swap}}

\newcommand{\colim}{\operatorname{colim}}

\newcommand{\undertext}[2]{\underbrace{#1}_{\text{#2}}}

\DeclareFontFamily{U}{musix}{}%
\DeclareFontShape{U}{musix}{m}{n}{%
  <-12>   musix11
  <12-15> musix13
  <15-18> musix16
  <18-23> musix20
  <23->   musix29
}{}%
% Not strictly necessary but convenient:
\newcommand*\musix{\usefont{U}{musix}{m}{n}\selectfont}
\DeclareTextFontCommand{\textmusix}{\musix}

\newcommand{\doubleflat}{{\raisebox{.6ex}{\textmusix{3}}}}
\newcommand{\doublesharp}{{\raisebox{.6ex}{\textmusix{5}}}}

% fibrations
\newcommand{\pull}{\mathsf{pull}}
\newcommand{\push}{\mathsf{push}}
\newcommand{\lift}{\mathsf{lift}}
\newcommand{\counitpull}{\lambda}
\newcommand{\unitpull}{\eta}

% arrow monad
\newcommand{\unitarrow}{\internal{\id}}
\newcommand{\multarrow}{\internal{\circ}}

% 2-cats
\newcommand{\DispSlice}[2]{\{{#1} \underset{\disp}{\twoto} {#2}\}}%{#1 /_{\!d}\, #2}

\newcommand{\comma}[2]{\{{#1} \twoto {#2}\}}
\newcommand{\isocomma}[2]{\{{#1} \isotwoto {#2}\}}
\newcommand{\pb}[2]{\{{#1} = {#2}\}}

\newcommand{\internal}[1]{\ulcorner{#1}\urcorner}

\newcommand{\generic}{\chi}
\newcommand{\isogeneric}{\widetilde{\chi}}

\newcommand{\walkingcomposite}{\boldsymbol{3}}
\newcommand{\walkingarrow}{\downarrow}%{{\boldsymbol{2}}}
\newcommand{\walkingobj}{\mathsf{pt}}%{{\boldsymbol{1}}}
\newcommand{\walkingloosearrow}{{\stackMath\mathrel{\stackinset{c}{0ex}{c}{0.05ex}{\scalebox{.6}{$\loosemarking$}}{\downarrow}}}}
\newcommand{\walkingsubobj}{\downsubseteq}

% dbl cats
\newcommand{\slackmarking}{\circ}%{\mathchoice{\circ}{\circ}{\scalebox{1.3}{$\circ$}}{\scalebox{1.3}{$\circ$}}}
\newcommand{\loosemarking}{\bullet}
\newcommand{\profmarking}{\shortmid}

\newcommand{\lcomp}{\odot}
\newcommand{\scomp}{\ocirc}
\newcommand{\tcomp}{}
\newcommand{\lid}{1}
\newcommand{\tid}{\id}

\newcommand{\llunitor}{\lambda}
\newcommand{\lrunitor}{\rho}
\newcommand{\lassociator}{\alpha}

\newcommand{\src}{s}
\newcommand{\tgt}{t}

\newcommand{\loose}{{\mathsf{loose}}}
\newcommand{\tight}{{\mathsf{tight}}}

\newcommand{\LooseSlice}[2]{\{{#1} \loosetwoto {#2}\}}%{{#1}\!\underset{\loose}{/}\!\!{#2}}
\newcommand{\RestrictedLooseSlice}[2]{\overline{\{{#1} \loosetwoto {#2}\}}}
\newcommand{\TightSlice}[2]{\{{#1} \twoto {#2}\}}%{{#1}\!\underset{\tight}{/}\!\!{#2}}

\newcommand{\SymMonDblCat}{{\dblcat{Sym}\dblcat{Mon}\DblCat}}

\newcommand{\One}{{\mathbf{1}}}
\newcommand{\mlunitor}{\ell}
\newcommand{\mrunitor}{r}
\newcommand{\massociator}{a}
\newcommand{\symmetry}{s}

\newcommand{\compositor}{\kappa}
\newcommand{\munitor}{\epsilon}
\newcommand{\mmultiplicator}{\nu}

% theory of systems
\newcommand{\unitor}{\eta}
\newcommand{\associator}{\mu}

\newcommand{\laxator}{\ell}

% systems theory
\newcommand{\compth}[1]{\dblcat{#1}}
\newcommand{\systh}[1]{\mathbf{#1}}
\newcommand{\doc}[1]{\mathfrak{#1}}

\newcommand{\Paradigm}{{(\dblcat{E}, T)}}
\newcommand{\Doc}{{\doc{Doc}}}
\newcommand{\Processes}{\dblcat{P}}
\newcommand{\Sys}{{\systh{Sys}}}
\newcommand{\Comp}{{\compth P}}

\newcommand{\lst}[1]{\underline{#1}}

\newcommand{\Cyb}{\systh{Cyb}}
\newcommand{\Cont}{\systh{Cont}}
\newcommand{\Plnt}{\systh{Plnt}}
\newcommand{\CybProcesses}{{\compth C}}
\newcommand{\CybComp}{{\trplcat C}}

\newcommand{\doctrine}{\mathfrak{D}}
\newcommand{\theory}{\mathbb{T}}
\newcommand{\CompTheories}{{\dblcat{Comp}\dblcat{Th}}}
\newcommand{\SysTheories}[1][{}]{{\dblcat{Sys}\dblcat{Th}_{#1}}}

\newcommand{\DblCat}{{\dblcat{Dbl}\Cat}}
\newcommand{\DblIx}{{\dblcat{Dbl}\dblcat{Ix}}}
\newcommand{\MonDblIx}{{\dblcat{Mon}\dblcat{Dbl}\dblcat{Ix}}}

\newcommand{\SymOperads}{{\sf sm}}
\newcommand{\SymDblOperads}{{\sf smc}}

\newcommand{\dblSys}{{\dblcat{Sys}}}

\newcommand{\unilaxto}{\nto{\text{unitary lax}}}
\newcommand{\theoryover}[1]{{#1}^\top \nto{\text{unitary lax}} \dblCat}

\newcommand{\sys}[1]{\mathsf{#1}}

\newcommand{\BehComp}{\dblSet}
\newcommand{\BehCybComp}{\trplSet}
\newcommand{\BehSys}{\BehComp^\walkingarrow}
\newcommand{\BehCyb}{\BehCybComp^\walkingarrow}

\newcommand{\Lens}{\dblcat{Lens}}
\newcommand{\ndLens}{\dblcat{Lens}_\pow}

\newcommand{\Moore}{\systh{Moore}}
\newcommand{\ndMoore}{\systh{Moore}_\pow}
\newcommand{\dblMoore}{\dblcat{Moore}}

\renewcommand{\Alph}{\dblcat{Alph}}
\newcommand{\Int}{\cat{Int}}
\newcommand{\Coalg}{\systh{Coalg}}
\newcommand{\Trans}{\systh{Trans}}
\newcommand{\DynSys}{\systh{DynSys}}
\newcommand{\FSM}{\systh{FSM}}
\newcommand{\Mealy}{\systh{Mealy}}
\newcommand{\Petri}{\systh{Petri}}
\newcommand{\ODE}{\systh{ODE}}
\newcommand{\Rel}{\dblcat{Rel}}

% behaviours
\newcommand{\Behaviour}{B}
\newcommand{\States}{X}
\newcommand{\Trajs}{J}
\newcommand{\Fixpts}{F}

\newcommand{\Blackbox}{\blacksquare}

\newcommand{\dblEnd}{\dblcat{End}}

\newcommand{\expose}{v}
\newcommand{\update}{v^\sharp}
% \newcommand{\observe}{\mathsf{observe}}
\newcommand{\Fix}{\sys{Fix}}
\newcommand{\Trivial}{{\sys{1}}}
\newcommand{\Clock}{{\sys{Clock}}}
\newcommand{\Time}{\sys{Time}}

\newcommand{\nreachto}[1]{\narrow{#1}{\rightsquigarrow}}

\newcommand{\Psh}{\cat{Psh}}

% diagrams
\newcommand{\LargeSquareDiagram}[9]{
	\begin{tikzcd}[ampersand replacement=\&]
		{#1} \&\& {#2} \\
		\\
		{#4} \&\& {#3}
		\arrow["{#5}", from=1-1, to=1-3]
		\arrow[""{name=0, anchor=center, inner sep=0}, "{#8}"', "\loosemarking"{marking}, from=1-1, to=3-1]
		\arrow[""{name=1, anchor=center, inner sep=0}, "{#6}", "\loosemarking"{marking}, from=1-3, to=3-3]
		\arrow["{#7}"', from=3-1, to=3-3]
		\arrow["{#9}", shorten <=13pt, shorten >=13pt, Rightarrow, from=0, to=1, shift right=0.75]
	\end{tikzcd}
}
\newcommand{\SmallSquareDiagram}[9]{
	\begin{tikzcd}[ampersand replacement=\&,cramped,sep=small]
		{#1} \&\& {#2} \\[-2pt]
		\\[-2pt]
		{#4} \&\& {#3}
		\arrow["{#5}", from=1-1, to=1-3]
		\arrow[""{name=0, anchor=center, inner sep=0}, "{#8}"', "\loosemarking"{marking}, from=1-1, to=3-1]
		\arrow[""{name=0p, anchor=center, inner sep=0}, phantom, from=1-1, to=3-1, start anchor=center, end anchor=center]
		\arrow[""{name=1, anchor=center, inner sep=0}, "{#6}", "\loosemarking"{marking}, from=1-3, to=3-3]
		\arrow[""{name=1p, anchor=center, inner sep=0}, phantom, from=1-3, to=3-3, start anchor=center, end anchor=center]
		\arrow["{#7}"', from=3-1, to=3-3]
		\arrow["{#9}", shorten <=8pt, shorten >=8pt, Rightarrow, from=0, to=1, shift right=0.75]
	\end{tikzcd}
}

% various constructions
\newcommand{\Span}{\dblcat{Span}}
\newcommand{\Cospan}{\dblcat{Cospan}}
\newcommand{\TriSpan}{\trplcat{Span}}
\newcommand{\FibSpan}{{\cat{f}\TriSpan}}
\newcommand{\OpfibSpan}{{\cat{o}\TriSpan}}
\newcommand{\PsCat}{{\trplcat{Ps}\trplcat{Cat}}}

\newcommand{\Ctx}{\trplcat{{Ctx}}}
\newcommand{\Eff}{\trplcat{{Eff}}}
\newcommand{\DispSpan}{{\cat{d}\TriSpan}}

\newcommand{\Comnd}{\dblcat{Comnd}}

\newcommand{\Para}{\dblcat{Para}}
\newcommand{\Copara}{\dblcat{Copara}}

\newcommand{\dblKl}{\dblcat{Kl}}
\newcommand{\dblAlg}{\dblcat{Alg}}

% fibred actions
\newcommand{\acted}{\dblcat C}
\newcommand{\actor}{\dblcat E}
\newcommand{\combine}{\mathbin{\otimes}}
\newcommand{\combineunit}{I}
\newcommand{\action}{\mathbin{\lhd}}

\newcommand{\lineator}{\lambda}
\newcommand{\colineator}{\kappa}

\newcommand{\scaledtriangle}{\mathchoice%
  {\scalebox{1.05}{$\triangle$}}%
  {\scalebox{1.1}{$\triangle$}}%
  {\scalebox{1.05}{$\triangle$}}%
  {\scalebox{1.03}{$\triangle$}}%
}
\newcommand{\mapunitor}{\eta}%{{\scaledtriangle}}
\newcommand{\mapmultiplicator}{\mu}%{{\pentagon}}

\newcommand{\counitor}{\varepsilon}
\newcommand{\comultiplicator}{\delta}
\newcommand{\coassociator}{\delta}

\newcommand{\monoidal}{\mathbin{\boxtimes}}
\newcommand{\monoidalunit}{\One}
\newcommand{\monoidalunitor}{\upsilon}
\newcommand{\monoidalassociator}{\alpha}

\newcommand{\monoidalinterch}{x}
\newcommand{\monoidalinterchunit}{u}

\newcommand{\leftunitlaw}{\lambda}
\newcommand{\rightunitlaw}{\rho}
\newcommand{\associativitylaw}{\alpha}

\newcommand{\spancomp}{\mathbin{\rotatebox[origin=c]{90}{$\ltimes$}}}
\newcommand{\isospancomp}{\mathbin{\dot{\rotatebox[origin=c]{90}{$\ltimes$}}}}

\newcommand{\downiso}{{\rotatebox[origin=c]{-90}{$\cong$}}}
\newcommand{\downsubseteq}{{\rotatebox[origin=c]{-90}{$\subseteq$}}}

\newcommand{\consdownarrows}{\begin{smallmatrix}\downarrow\\[-0.2ex]\downarrow\end{smallmatrix}}

% completions and wreath product
\newcommand{\Mnd}{\dblcat{Mnd}}%{\trplcat{KL}}
\newcommand{\KL}{\mathsf{KL}}%{\trplcat{KL}}
\newcommand{\EM}{\mathsf{EM}}%{\trplcat{EM}}
\newcommand{\wreath}{\wr}

% restriction to functorial spans
\newcommand{\fun}{{\mathsf{fun}}}
\newcommand{\opfun}{{\mathsf{opfun}}}

% adjunctions
\newcommand{\unit}{\eta}
\newcommand{\counit}{\epsilon}

% monads
\newcommand{\monad}{s}
\newcommand{\unitmnd}{\upsilon}
\newcommand{\multmnd}{\mu}

\newcommand{\intertwiner}{w}

\newcommand{\Cosmos}{\Kb}
\newcommand{\Display}{\Da}
\newcommand{\Paradise}{(\Cosmos,\Display)}
\newcommand{\DualParadise}{(\Cosmos\co,\Display)}

% commas & isocommas
\newcommand{\InclusionTrifun}{(-,\, (-)^\downarrow)}%{I}
\newcommand{\DualInclusionTrifun}{(-,\, (-)^\uparrow)}%{I}

\newcommand{\colaxity}[1]{\overline{#1}}

\newcommand{\simple}{\mathsf{Simp}}
\newcommand{\subm}{\mathsf{Subm}}
\newcommand{\Subm}{\Smooth^\backdown}

% colax fibred actions
\newcommand{\CxFibAct}{\trplcat{Cx}\trplcat{Fib}\trplcat{Act}}
\newcommand{\LxOpfibAct}{\trplcat{Lx}\trplcat{Opfib}\trplcat{Act}}

\newcommand{\extch}{\mathbin{\&}}

% fact system
\newcommand{\leftclass}{{\La}}
\newcommand{\rightclass}{{\Ra}}

\newcommand{\quintets}[1]{{\dblcat{Q}(#1)}}
\newcommand{\looseleftto}[1][]{\begin{tikzcd}[ampersand replacement=\&,cramped,sep=small]
  {} \&[1ex] {}
  \arrow["{#1}", "\shortmid"{marking,pos=0.4}, two heads, from=1-1, to=1-2]
\end{tikzcd}}
\newcommand{\looserightto}[1][]{\begin{tikzcd}[ampersand replacement=\&,cramped,sep=small]
  {} \&[1ex] {}
  \arrow["{#1}", "\shortmid"{marking}, tail, from=1-1, to=1-2]
\end{tikzcd}}

% doctrines of wreaths
\newcommand{\AdTrpl}{{\trplcat{Ad}\trplcat{Trpl}}}

\newcommand{\forwto}{\rightarrowtail}
\newcommand{\backto}{\twoheadrightarrow}
\newcommand{\forwdown}{{\rotatebox[origin=c]{-90}{$\rightarrowtail$}}}
\newcommand{\backdown}{{\rotatebox[origin=c]{-90}{$\twoheadrightarrow$}}}

\newcommand{\FunWreaths}{\trplcat{Wrt}_\fun}
\newcommand{\Ctxad}{\trplcat{Ctd}}

% synthetic pullback operation
\newcommand{\tow}[1]{#1 \cdot}
\newcommand{\combineunitcart}{\upsilon}
\newcommand{\combinecart}{\kappa}

\newcommand{\LooseArrow}[1]{{#1}^\walkingloosearrow}
% \newcommand{\RestrictedLooseArrow}[1]{\overline{#1}^\walkingloosearrow}

\newcommand{\Cubes}[1]{{#1}^{\scalebox{.75}{\mancube}}}
% \newcommand{\Ravioli}[1]{{#1}^{ravioli}}

\newcommand{\simp}{\ltimes}
\newcommand{\sel}{\mathsf{sel}}

\newcommand{\image}{\operatorname{im}}
\newcommand{\Prod}{\cat{Prod}}

\newcommand{\etale}{\mathsf{ét}}
\newcommand{\React}{\mathrm{React}}
\newcommand{\ErgKit}{\systh{ErgKit}}
\newcommand{\ErgSys}{\systh{Erg}}
\newcommand{\dblSmooth}{\dblcat{Smooth}}
\newcommand{\dblEtale}{\dblcat{{\acute{E}}tale}}
\newcommand{\ReactSys}{\systh{React}}

\newcommand{\Goals}{\systh{Goals}}
\newcommand{\Pred}{\systh{Pred}}
\newcommand{\IntGoals}{\cat{IntGoals}}
\newcommand{\PerfReg}{\cat{PerfReg}}

\newcommand{\GoalsFib}{\pi}
\newcommand{\Sh}{\cat{Sh}}

\newcommand{\fst}{\mathsf{fst}}

\newcommand{\states}{\mathsf{states}}
\newcommand{\trajectories}{\mathsf{trajs}}

\newcommand{\always}{\square}
\newcommand{\eventually}{\diamondsuit}

\newcommand{\SysTh}{\dblcat{Sys}\dblcat{Th}}

\newcommand{\sem}[1]{\llbracket{#1}\rrbracket}
\newcommand{\then}{\fatsemi}

\newcommand{\entails}{\vdash}
\newcommand{\sementails}{\vDash}

\newcommand{\nleadsto}[1]{\overset{#1}{\leadsto}}
