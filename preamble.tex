\usepackage{todonotes}
\usepackage{fonttable}
\usepackage{stmaryrd}
\usepackage{wasysym}
\usepackage{tabularx}
\usepackage{expl3}
\usepackage{xparse, xpatch}
\usepackage{stackengine, old-arrows}

\usepackage{quiver}
\usepackage{tangle}

% todos
\newcommand{\matteo}[1]{\todo[inline,color=green!30]{\textbf{Matteo}: {#1}}}

% scaled & centered figures
\newcommand{\sctikzfig}[2][.8]{\begin{center}\scalebox{#1}{\tikzfig{#2}}\end{center}}

% aligned equations
\newenvironment{eqalign}{\begin{equation}\begin{aligned}}{\end{aligned}\end{equation}}
\newenvironment{eqalign*}{\begin{equation*}\begin{aligned}}{\end{aligned}\end{equation*}}

% wide equations (for eqs in enums)
\newenvironment{eqwide}{\begin{equation}\displayindent0pt\displaywidth\textwidth}{\end{equation}}
\newenvironment{eqalignwide}{\begin{eqalign}\displayindent0pt\displaywidth\textwidth}{\end{eqalign}}


\tikzset{
  relation/.style={
    draw=none,
    every to/.append style={
      edge node={node [sloped, allow upside down, auto=false]{$#1$}}}
  }
}

% common arrow styles
\tikzcdset{
  mono/.code={
    \pgfsetarrows{tikzcd to reversed-tikzcd to}
  }
}
\tikzcdset{
  epi/.code={
    \pgfsetarrowsend{tikzcd double to}
  }
}
\tikzcdset{
  into/.code={
    \pgfsetarrows{tikzcd right hook-tikzcd to}
  }
}
\tikzcdset{
  twocell/.style={Rightarrow, shorten >= 3ex, shorten <= 3ex}
}

\tikzcdset{
  row sep/normal={
    6ex
  },
  column sep/normal={
    8ex
  }
}

% comment on an equation
\newcommand{\comment}[1]{\qquad\text{#1}}

% disjoint footnotes
\newcommand{\disjointfootnotemark}[1]{\footnotemark[\getrefnumber{#1}]}
\newcommand{\disjointfootnotetext}[1]{%
  \addtocounter{footnote}{1}%
  \addtocounter{Hfootnote}{1}%
  \footnotetext{#1}%
}

% overset without decreasing font size
\newcommand{\Overset}[2]{%
  \mathop{#2}\limits^{\vbox to -.1ex{%
  \kern -1.8ex\hbox{$#1$}\vss}}%
}
% underset without decreasing font size
\newcommand{\Underset}[2]{%
  \mathop{#2}\limits_{\vbox to .1ex{%
  \kern -.6ex\hbox{$#1$}\vss}}%
}

% fat semicolon
\newcommand{\comp}{\fatsemi}

% hyphen for math mode
\mathchardef\dash="2D

% defined term
\newcommand{\defining}[1]{\textbf{#1}}

% subject of a thesis
\renewcommand{\th}[1]{\overset{th}{#1}}

% e costant
\newcommand{\e}{\mathrm{e}}

% exp
\renewcommand{\exp}{\operatorname{exp}}

% cotangent
\newcommand{\cotan}{\operatorname{cotan}}

% argoptim
\DeclareMathOperator{\argmin}{\mathrm{argmin}}
\DeclareMathOperator{\argmax}{\mathrm{argmax}}

% 'does not imply' symbol
\newcommand{\nimplies}{\centernot\implies}

% implications in the opposite direction
\newcommand{\implied}{\Longleftarrow}
\newcommand{\nimplied}{\centernot\implied}

% logical implication
\newcommand{\limp}{\rightarrow}
\newcommand{\liff}{\leftrightarrow}

% inhabitation for types
\newcommand{\inh}{\!\colon\!}

% inverses of \to
\newcommand{\from}{\leftarrow}

% long version of \to
\newcommand{\longto}{\longrightarrow}

% inverse of \mapsto
% \newcommand{\mapsfrom}{\mathrel{\reflectbox{\ensuremath{\mapsto}}}}
% \newcommand{\longmapsfrom}{\mathrel{\reflectbox{\ensuremath{\longmapsto}}}}

% inclusion
\newcommand{\into}{\hookrightarrow}
\newcommand{\infrom}{\hookleftarrow}
\newcommand{\monoto}{\rightarrowtail}
\newcommand{\monofrom}{\leftarrowtail}

% surjection
\newcommand{\onto}{\twoheadrightarrow}
\newcommand{\epito}{\twoheadrightarrow}
\newcommand{\epifrom}{\twoheadleftarrow}

% 2-morphisms
\newcommand{\twoto}{\Rightarrow}
\newcommand{\isotwoto}{\overset{\sim}\twoto}
\newcommand{\longtwoto}{\Longrightarrow}
\newcommand{\isolongtwoto}{\overset{\sim}\longtwoto}

% 3-morphisms
\newcommand{\threeto}{\Rrightarrow}

\newcommand{\narrow}[2]{\overset{#1}{#2}}
\newcommand{\nto}[1]{\xrightarrow{#1}}
\newcommand{\nlongto}[1]{\xrightarrow{\;#1\;}}
\newcommand{\ninto}[1]{\narrow{#1}{\into}}
\newcommand{\nepito}[1]{\narrow{#1}{\epito}}
\newcommand{\nmonoto}[1]{\narrow{#1}{\monoto}}

\newcommand{\nfrom}[1]{\narrow{#1}{\from}}
\newcommand{\nlongfrom}[1]{\xleftarrow{\;#1\;}}
\newcommand{\ninfrom}[1]{\narrow{#1}{\infrom}}
\newcommand{\nepifrom}[1]{\narrow{#1}{\epifrom}}
\newcommand{\nmonofrom}[1]{\narrow{#1}{\monofrom}}

\newcommand{\nisoto}[1]{\xrightarrow[\sim]{#1}}
\newcommand{\nisofrom}[1]{\xleftarrow[\sim]{#1}}
\newcommand{\isofrom}[1]{\nisofrom{}}
\newcommand{\isoto}[1]{\nisoto{}}

% profunctors
\newcommand{\profto}{\stackMath\mathrel{\stackinset{c}{-0.25ex}{c}{0.25ex}{\shortmid}{\to}}}
\newcommand{\longprofto}{\stackMath\mathrel{\stackinset{c}{-0.25ex}{c}{0.25ex}{\shortmid}{\longto}}}
\newcommand{\nprofto}[1]{\narrow{#1}{\profto}}

% double categories
\newcommand{\horto}{\longto}
\newcommand{\verto}{\stackMath\mathrel{\stackinset{c}{0ex}{c}{0.05ex}{\bullet}{\longto}}}
\newcommand{\nhorto}[1]{\narrow{#1}{\horto}}
\newcommand{\nverto}[1]{\narrow{#1}{\verto}}

% optics
\newcommand{\opticto}{\leftrightarrows}
\newcommand{\chartto}{\rightrightarrows}
\newcommand{\equalto}{=\mathrel{\mkern-3mu}=}
\newcommand{\nequalto}[1]{\overset{#1}{\equalto}}
\newcommand{\nopticto}[2]{\overset{#1}{\underset{#2}\opticto}}
\newcommand{\nchartto}[2]{\overset{#1}{\underset{#2}\chartto}}

% 'a | b'
\newcommand{\divides}{\,|\,}

% constant function
\newcommand{\cost}{\text{cost.}}
\newcommand{\const}{\mathsf{const}}

% locutions
\newcommand{\word}[1]{\quad\text{\underline{#1}}\quad}
\newcommand{\almosteverywhereon}[2][\mu]{{\text{${#1}$-a.e. on ${#2}$}}}
\renewcommand{\ae}{\ \text{a.e.}}
\newcommand{\sse}{\word{iff}}
\newcommand{\means}{\word{means}}
\newcommand{\impl}{\word{implies}}
\newcommand{\fa}{\ \text{f.a.}\;}

% such that
\newcommand{\suchthat}{\,|\,}

% numerical sets
\newcommand{\N}{\mathbb{N}}
\newcommand{\Z}{\mathbb{Z}}
\newcommand{\Q}{\mathbb{Q}}
\newcommand{\R}{\mathbb{R}}
\newcommand{\C}{\mathbb{C}}

% set-theoretic stuff
\newcommand{\card}[1]{\left|{#1}\right|}
\newcommand{\parts}[1]{\mathcal{P}\left({#1}\right)}
\newcommand{\continuum}{\mathfrak{c}}

% diameter of a set
\newcommand{\diam}{\operatorname{diam}}

% vectors
\newcommand{\vers}[1]{\hat{\vv{#1}}}

\newcommand{\ii}{\vers{i}}
\newcommand{\jj}{\vers{j}}
\newcommand{\kk}{\vers{k}}

\newcommand{\xx}{\vv{x}}
\newcommand{\yy}{\vv{y}}
\newcommand{\zz}{\vv{z}}

% big kernel, cokernel & image
\newcommand{\Ker}{\operatorname{Ker}}
\newcommand{\coker}{\operatorname{coker}}
\newcommand{\Imm}{\operatorname{Im}}
\newcommand{\im}{\operatorname{im}}

% action of a group
\newcommand{\acts}{\curvearrowright}
% weak action groupoid
\newcommand{\wag}{\mathbin{/\mkern-6mu/}}

% direct sum
\newcommand{\dir}{\oplus}
\newcommand{\bigdir}{\bigoplus}

% operations in an Heyting algebra
\newcommand{\hey}{\Rightarrow}
\newcommand{\bigsup}{\bigvee}
\newcommand{\biginf}{\bigwedge}

% differential
\newcommand{\diff}[1]{\operatorname{d}{#1}}
% jacobian
\newcommand{\jac}{\operatorname{\vv{J}}}

% derivatives
\newcommand{\de}{\mathrm{d}}
\newcommand{\dx}{\de x}
\newcommand{\dt}{\de t}
\newcommand{\ds}{\de s}

\newcommand{\der}[2]{\frac{\de{#1}}{\de{#2}}}
\newcommand{\pder}[2]{\frac{\partial {#1}}{\partial {#2}}}

% second derivatives
\newcommand{\sder}[2]{\frac{\de^2{#1}}{\de{#2}^2}}
\newcommand{\spder}[3]{\frac{\partial^2{#1}}{\partial{#2} \partial{#3}}}
% second derivative on the same coordinate
\newcommand{\sdpder}[2]{\frac{\partial^2{#1}}{\partial{#2}^2}}

% big derivatives
\newcommand{\bigder}[2]{\dfrac{\strut \de{#1}}{\de{#2}}}
\newcommand{\bigpder}[2]{\dfrac{\strut \partial {#1}}{\partial {#2}}}

% big second derivatives
\newcommand{\bigsder}[2]{\dfrac{\strut \de^2 {#1}}{\de{#2}^2}}
\newcommand{\bigspder}[3]{\dfrac{\strut \partial^2 {#1}}{\partial {#2} \partial {#3}}}
% big second derivative on the same coordinate
\newcommand{\bigsdpder}[2]{\dfrac{\strut \partial^2 {#1}}{\partial {#2}^2}}

% left/right applied partial derivatives
\newcommand{\lpartial}{\overset{\leftarrow}\partial}
\newcommand{\rpartial}{\overset{\rightarrow}\partial}

% complex stuff
\newcommand{\conj}[1]{\overline{#1}}
\newcommand{\Arg}{\operatorname{Arg}}
\newcommand{\Res}{\operatorname{Res}}

% real and imaginary parts
\renewcommand{\Re}[1]{\mathfrak{Re}\left(#1\right)}
\renewcommand{\Im}[1]{\mathfrak{Im}\left(#1\right)}

% sign function
\newcommand{\sign}{\operatorname{}{sgn}}

% convergence
\newcommand{\conv}[1][]{\underset{{#1}}{\longrightarrow}}

% regularity classes
\newcommand{\Cn}{\mathcal{C}}
\newcommand{\Czero}{\Cn^0}
\newcommand{\Cone}{\Cn^1}
\newcommand{\Ctwo}{\Cn^2}
\newcommand{\Cinfty}{\Cn^\infty}

% Lipschitz
\newcommand{\Lip}{\mathrm{Lip}}

% indicator function
\newcommand{\ind}{\vv{1}}

% lenses
\newcommand{\biglens}[2]{
	 \begin{pmatrix}{\vphantom{f_f^f}#1} \\ {\vphantom{f_f^f}#2} \end{pmatrix}
}
\newcommand{\littlelens}[2]{
	 \begin{psmallmatrix}{\vphantom{f}#1} \\ {\vphantom{f}#2} \end{psmallmatrix}
}
\newcommand{\lens}[2]{
  \relax\if@display
	 \biglens{#1}{#2}
  \else
	 \littlelens{#1}{#2}
  \fi
}

\usepackage{xstring}
\newcommand{\cat}[1]{
  \relax
  \StrLen{#1}[\catarglen]
  \ifnum\catarglen=1
    \mathcal{#1}
  \else
    \mathbf{#1}
  \fi
}
\newcommand{\dblcat}[1]{\cat{\mathbb #1}}
\newcommand{\trplcat}[1]{\cat{\mathfrak #1}}

\newcommand{\dblSet}{\dblcat{Set}}
\newcommand{\dblCat}{\dblcat{Cat}}

\newcommand{\Poly}{\cat{Poly}}

\newcommand{\dist}{\Delta}
\newcommand{\pow}{\mathcal{P}}

\newcommand{\cod}{\mathrm{cod}}
\newcommand{\dom}{\mathrm{dom}}

\newcommand{\st}{\mathrm{st}}

\newcommand{\Sub}{\mathrm{Sub}}

\newcommand{\eval}{\mathrm{eval}}
\newcommand{\curr}{\mathrm{curr}}

\newcommand{\true}{\mathsf{true}}

\newcommand{\view}{\mathsf{view}}
\newcommand{\play}{\mathsf{play}}
\newcommand{\coplay}{\mathsf{coplay}}

\newcommand{\name}[1]{\lceil #1 \rceil}

% identity
\newcommand{\identity}{\mathrm{id}}
\newcommand{\id}{\mathrm{id}}

% isomorphism and equivalence symbols
\newcommand{\iso}[1][]{\overset{#1}{\cong}}
\newcommand{\equi}{\simeq}

% F left adjoint to G symbol
\newcommand{\adj}{\dashv}

% categories
\newcommand{\Ob}{\operatorname{Ob}}
\newcommand{\Hom}{\operatorname{Hom}}
\newcommand{\End}{\operatorname{End}}
\newcommand{\Aut}{\operatorname{Aut}}
\newcommand{\Nat}{\operatorname{Nat}}

% Kan extensions
\newcommand{\Lan}{\operatorname{Lan}}
\newcommand{\Ran}{\operatorname{Ran}}

% big categories
\newcommand{\Cat}{\cat{Cat}}
\newcommand{\Prof}{\cat{Prof}}

\newcommand{\Set}{\cat{Set}}
\newcommand{\FinSet}{\cat{FinSet}}

\newcommand{\Mon}{\cat{Mon}}
\newcommand{\CMon}{\cat{CMon}}
\newcommand{\Grp}{\cat{Grp}}
\newcommand{\Mod}{\cat{Mod}}
\newcommand{\Ab}{\cat{Ab}}
\newcommand{\Vect}{\cat{Vect}}
\newcommand{\Met}{\cat{Met}}
\newcommand{\Meas}{\cat{Meas}}
\newcommand{\Msbl}{\cat{Msbl}}
\newcommand{\Prob}{\cat{Prob}}
\newcommand{\Euc}{\cat{Euc}}
\newcommand{\Smooth}{\cat{Smooth}}

% opposite category
\newcommand{\op}{\mathsf{op}}
\newcommand{\co}{\mathsf{co}}
\newcommand{\coop}{\mathsf{coop}}

\newcommand{\Span}{\cat{Span}}
\newcommand{\Para}{\cat{Para}}
\newcommand{\Copara}{\cat{Copara}}
\newcommand{\Optic}{\cat{Optic}}
\newcommand{\Lens}{\cat{Lens}}
\newcommand{\DLens}{\cat{DLens}}
\newcommand{\DChart}{\cat{DChart}}

\newcommand{\Bun}{\mathrm{Bun}}

\newcommand{\MonCat}{\cat{MonCat}}
\newcommand{\SymMonCat}{\cat{SymMonCat}}
\newcommand{\Fib}{\cat{Fib}}
\newcommand{\OpFib}{\cat{OpFib}}
\newcommand{\Kl}{\cat{Kl}}
\newcommand{\coKl}{\cat{coKl}}
\newcommand{\biKl}{\cat{biKl}}

\newcommand{\Alg}[1]{{#1}\dash\cat{Alg}}
\newcommand{\Coalg}[1]{{#1}\dash\cat{Coalg}}
\newcommand{\Bialg}[2]{({#1},{#2})\dash\cat{BiAlg}}

\newcommand{\lax}{\mathrm{lx}}
\newcommand{\oplax}{\mathrm{ox}}
\newcommand{\pseudo}{\mathrm{ps}}
\newcommand{\strict}{\mathrm{s}}
\newcommand{\cart}{\mathrm{cart}}
\newcommand{\ver}{\mathrm{vert}}

\newcommand{\VCat}[1]{{#1}\dash\Cat}

\newcommand{\rev}{\mathrm{rev}}
\newcommand{\swap}{\mathrm{swap}}

\newcommand{\colim}{\operatorname{colim}}

\newcommand{\undertext}[2]{\underbrace{#1}_{\text{#2}}}

\DeclareFontFamily{U}{musix}{}%
\DeclareFontShape{U}{musix}{m}{n}{%
  <-12>   musix11
  <12-15> musix13
  <15-18> musix16
  <18-23> musix20
  <23->   musix29
}{}%
% Not strictly necessary but convenient:
\newcommand*\musix{\usefont{U}{musix}{m}{n}\selectfont}
\DeclareTextFontCommand{\textmusix}{\musix}

\newcommand{\doubleflat}{{\raisebox{.6ex}{\textmusix{3}}}}
\newcommand{\doublesharp}{{\raisebox{.6ex}{\textmusix{5}}}}

\newcommand{\dblSpan}{\dblcat{Span}}
\newcommand{\DblCat}{\dblcat{DblCat}}
\newcommand{\DblIx}{\dblcat{Dbl}\dblcat{Ix}}
\newcommand{\MonDblIx}{\dblcat{Mon}\dblcat{Dbl}\dblcat{Ix}}

\newcommand{\sys}{\mathsf}
\newcommand{\systh}{\mathbf}

\newcommand{\Processes}{\dblcat{P}}
\newcommand{\CyberProcesses}{\trplcat{P}}

\newcommand{\Sys}{\systh{Sys}}
\newcommand{\Cyb}{\cat{Cyb}}
\newcommand{\CybSys}{\cat{CybSys}}

\newcommand{\doctrine}{\mathfrak{D}}
\newcommand{\theory}{\mathbb{T}}
\newcommand{\Theories}{\dblcat{Theories}}
\newcommand{\Behaviour}{\cat{B}}

\newcommand{\Obs}{\systh{Obs}}
\newcommand{\Moore}{\systh{Moore}}
